%/////////////////////////////////////////////////////////
% 未踏 提出書類 様式4
% title  : "代替現実に向けたインタラクティブなデバイスの開発"
% file   : main.tex
% date   : 2018/02/28 Keita Mizushima
%
%/////////////////////////////////////////////////////////
\documentclass[11pt,onecolumn]{jsarticle}
%\documentclass[11pt,a4j,onecolumn]{jsarticle}
\usepackage[dvipdfmx]{color,graphicx}
\usepackage{amsmath}
\usepackage{txfonts}
\usepackage{booktabs}
\usepackage[dvipdfmx]{graphicx}
\usepackage{afterpage}

\begin{document}
\begin{flushright}
【様式4】
\end{flushright}

\begin{flushleft}
テーマ名:代替現実に向けたインタラクティブなデバイスの開発\\
申請者名:水島啓太、櫻井優真
\end{flushleft}
\begin{center}
【提案テーマ詳細説明】
\end{center}

\section{なにをつくるか}
現在コンピュータ世界への入力手法としてもっとも広く利用されているのは
キーボードである。現在、主流となっているキーボードの配列はQWERTY配列と呼ばれ、1872年にクリストファー・レイサム・ショールズによって
原型がつくられた。この配列の理由として諸説あるが、一番有名な説は「タイプライター
の機構が貧弱で、あまり早く打ち込むと動作不良を起こすため、適度な速度でタイピング
できるように調節した」というものである。
また異なるキーボード配列としてDvorak配列がある。
この配列は1932年にオーガスト・ドヴォラックによって考案されたアルファベットの出現頻度や打ちやすさを考慮したキー配置となっている。
しかし、使用に際して大きな差は実感できなかった。\\
\\

今後、我々は人工現実時代に到達すると考えられるが、この世界でもキーボードは使われるのだろうか。
この先、IC性能の向上とともにコンピュータというものはほとんどその実在を感じないものになっていくと考える。
50年先から今を見れば、なぜスマートフォンやノートパソコン、タブレットのような重い端末を持ち歩いていたのか
不思議に思うことだろう。端末が進化するなら入力方法もそれに適したものである必要がある。
そのため、我々はここでコンピュータ世界への入力デバイスを人工現実において再発明する必要があると考える。
\\
また、人工現実時代のデバイスに求められる機能はこれだけではないだろう。
我々はこれらのデバイスに人間の五感に働きかけるような作用が必要だと考える。
しかし、既存のVRやARといったような仮想現実を謳う技術では視覚的、聴覚的な感覚しか得ることができず仮想空間上に出力される物体から質感を感じ取ることはできない。
そこで装着型デバイスを用いてコンピュータ世界への入力を可能にしつつ、皮膚に圧力や振動、温度変化などを与えて仮想空間上の物体に
擬似的な質感を与え,第二の現実空間として代替現実を創り出すことを目的とするインタラクティブなデバイスの開発を行う。
映像として出力されたものは当然ながら質量がないため触れることはできない。しかし、触れることはできなくてもデバイスを用いて触れた感覚を与えることは
可能ではないだろうか。
%人の皮膚は振動数や圧力、温度変化を感知する触覚受容器により触れたという感覚を得ている。

人間の感覚は大きく分けると力覚と触覚を司る体性感覚,内蔵感覚,特殊感覚からなっている.
今回我々は手に装着する形のデバイスを考えており,この点から体性感覚に注目した.
には,振動や手足の曲げ伸ばしの程度を自覚する感覚や,物に触れた際の反力を感じる感覚,つまり力覚を指す.
一方で皮膚感覚は,皮膚表面に存在する受容器によって感じ取る感覚で,痛みや温度・凹凸・摩擦
などの触覚を指す[2].力覚を提示するデバイスとして開発された,ペン型のPHANToMやグ
ローブ型のCyberGrasp[4] は,様々な物質の物理モデルでの利用が進み,手術手技の模擬や
訓練などに応用されている.一方で皮膚感覚を提示するデバイスの応用として,インター
ネットによる通信販売で服の生地の特徴の伝達[11],タッチパネルの操作感覚の明確化,手術支援
システム[12],盲目者への視覚代行などが挙げられるが,空間形状・凹凸・温度・摩擦な
ど特定の感覚を提示するものに留まっている.手法の多さからも説明できるように,皮膚
感覚の提示の難しさは感覚の多様性にある

%私は人工現実時代においてコンピュータというものはほとんどその実在を感じないものになっていく
%と考える。今の我々は空気のさほど意識することがないように、コンピュータを日常的に意識することはほとんどないだろう。
%そのような世界において求められるデバイスとは何か。
%私はそのデバイスを新しい皮膚のようなものだと考える。それは単なる皮膚ではなく
%コンピュータ世界において機能する新しい皮膚だ。
%これを用いることで我々はデジタル空間に対する触覚を獲得し、コンピュータ世界と
%現実世界との境目をあいまいにできると考える。

\subsection{備えている機能}
本デバイスの構成要素として我々が考えているのは以下の要素である。
\begin{itemize}
\item 電気信号を用いてデジタル物体に実体を与える\\
電気生理学の知見から電気によって人間の運動を生成することができることが知られている。
電気信号を用いてコンピュータ世界に表示した物体に対して、あたかも触っているような
感覚を与えることが可能だと考える。物体にふれたと判断すれば、触れた強度に比例した
電気信号を流すことで触れたと錯覚させることができるものと考える。

\item コンピュータ世界の実現にむけた入力機構\\
人工現実時代においてコンピュータはどこにでも存在するものであると考える。
今ある例でいえばグーグルグラスなどがこれにあたる。グーグルグラスを用いる
ことで場所を選ばずコンピュータ世界に触れることができる。
このような世界に適した入力方式を考える必要がある。
今我々が用いている打鍵式の入力方式ではなく、スワイプやタップなどを組み合わせた
新たな入力方式を提案する。

\item 電気信号による運動の獲得\\
先に述べた通り電気信号を用いることで人間の運動を制御することができる。
運動の電気信号をストレージにためておくことで自分が学びたい動きを
その場で学ぶことができる。例えば、人気絵師のような絵を描きたいという
時に従来であれば模写から始め、長いトレーニングの末ようやく書くことができる。
これは絵というものが強い主観性を持っているからだと考える。絵のタッチというものは
なかなか人には説明できない。僕は本をみて勉強しても全く描けるようにならなかった。

このような例において電気信号による運動の獲得は大きな役割を果たす。
人気絵師が絵をかくときの電気信号をストレージにためておき、それを
こちらのデバイスで再現することで細かなペンのタッチや描くスピードなど
それまでは決して得ることのできなかった情報を多くの人が獲得できる
ようになると考える。

\item 電気信号を用いた触覚の再現\\
また従来の研究から電気信号を用いて触覚を再現することもできるという結果もある。
これはできたら、という思いが強いが、電気信号を用いてある程度の触覚を再現したい。
\end{itemize}

かくこと
\begin{itemize}
 \item 電気信号を用いた運動獲得とその例(絵、運動)
 \item 電気信号を用いた触覚の再現
 \item キーボードに変わる入力方法
 \item 神経系の構造
 \item デバイスの構造
 \item 用いる環境
 \item 備えている機能
\end{itemize}

\section{どんな出し方を考えているか}
このデバイスは人工現実に向けた代替現実を実現するためのものである。
さらにこのデバイスでは自分の動きを(絵を書くなど)センサを用いて抽出し、
このデータをクラウド上にためておくことで

\section{斬新さの主張、期待される効果など}
人工現実時代の新しい入力方式
動きをストレージに格納しておくことで、絵などの手を使った作業を簡単に獲得できる。
(これには再現性がある。電気信号なしでもその動きをできるようになる)



\section{具体的な進め方と予算}
我々は学生のため、開発時間は

\section{提案者(たち)の腕前を証明できるもの}
\[
  \underline{水島啓太}
\]
プログラミング言語:c,c++,java,python\\
専門分野:CPG,ディープラーニング、機械学習\\
その他スキル:Unityによる開発、画像処理、音声認識,物理モデル構築,動力学シミュレータchoreonoidでのシミュレーション\\
TOEIC 725点,

\[
  \underline{櫻井優真}
\]



\section{プロジェクト遂行にあてっての特記事項}
\[
  \underline{水島啓太}
\]\\
津山高専を卒業後,専攻科に進学予定です.しかし,来年度他大学への編入学を検討しており合否によって自主退学するかもしれません.

\[
  \underline{櫻井優真}
\]\\

\section{ソフトウェア開発以外の勉強、特技、生活、趣味など}
\section{将来のソフトウェア技術について思うこと・期すること}

%%%%%%%%%%%%%%%%参考文献%%%%%%%%%%%%%%%%%%%%
%\begin{thebibliography}{9}
%{\small

%}
%\end{thebibliography}
%%%%%%%%%%%%%%%%ここまで%%%%%%%%%%%%%%%%%%%%
\end{document}
